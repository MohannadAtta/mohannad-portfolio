\documentclass[10.5pt,a4paper]{article}

% ================== PACKAGES ==================
\usepackage[utf8]{inputenc}
\usepackage[T1]{fontenc}
\usepackage[ngerman]{babel}
\usepackage{geometry}
\usepackage{graphicx}
\usepackage{xcolor}
\usepackage{tabularx}
\usepackage{enumitem}
\usepackage{hyperref}
\usepackage{array}
\usepackage{qrcode}

% ================== PAGE LAYOUT ==================
\geometry{
  top=1.8cm,
  bottom=2.0cm,
  left=1.8cm,
  right=1.8cm
}

\setlength{\parindent}{0pt}
\setlength{\parskip}{2pt}
\renewcommand{\familydefault}{\sfdefault}

\definecolor{lightgray}{gray}{0.90}

% Compact bullet lists
\setlist[itemize]{leftmargin=1.5em,itemsep=2pt,topsep=2pt}

% Section heading
\newcommand{\cvsection}[1]{%
  \vspace{0.25cm}%
  {\bfseries #1}\par
  \vspace{0.15cm}%
  \rule{\textwidth}{0.35pt}\vspace{0.15cm}%
}

\begin{document}

% ================== HEADER ==================
\begin{tabularx}{\textwidth}{@{}>{\centering\arraybackslash}p{3cm} X@{}}
  \raisebox{-0.7\height}{\includegraphics[width=3cm,height=3.6cm,keepaspectratio]{Mohannad_Photo}}
  &
  \begin{minipage}[c]{\linewidth}
      \begin{center}
          {\Large \textbf{Mohannad Atta}}\\[2pt]
          \rule{\linewidth}{0.4pt}\\[3pt]
      \end{center}
      \colorbox{lightgray}{%
        \parbox{\linewidth}{\small
        \textbf{Adresse:} Erlangen, Germany \hfill
        \textbf{Mobil:} +49\,1573\,8174858\\
        \textbf{Email:} \href{mailto:mohannadatta@hotmail.com}{mohannadatta@hotmail.com} \hfill
        \textbf{LinkedIn:} \href{https://www.linkedin.com/in/mohannad-atta/}{LinkedIn Profil}\\
        \textbf{Website:} \href{https://mohannadatta.github.io/mohannad-portfolio/}{mohannadatta.github.io/mohannad-portfolio}\\[0.1cm]
        \hfill\qrcode[height=1cm]{https://mohannadatta.github.io/mohannad-portfolio/}
        }
      }
  \end{minipage}
\end{tabularx}

\vspace{0.4cm}

% ================== PROFIL ==================
\cvsection{PROFIL}

Ambitionierter MSc-Student der \textbf{Autonomy Technologies (FAU)} und aktueller \textbf{CTO von Earlybuild GmbH}, mit fundierter Erfahrung in autonomen Systemen, industrieller Automatisierung, Robotik, Embedded Systems und Softwareentwicklung.

Beherrscht moderne Technologien wie \textbf{Python, C++, ROS2, MATLAB/Simulink, Flutter, Firebase}, Cloud-Architekturen und KI-gestützte Datenanalyse.

Stark in Projektmanagement, Systemdesign und cross-funktionaler Zusammenarbeit. Leidenschaftlich daran interessiert, innovative, skalierbare und zukunftsorientierte Lösungen im Kontext von \textbf{Industrie 4.0, Robotik, Automatisierung und autonomen Systemen} zu entwickeln.

% ================== AKTUELLE POSITION ==================
\cvsection{AKTUELLE POSITION}

\textbf{CTO \& Software Engineer – Earlybuild GmbH} \hfill \textit{seit Okt. 2024}

\begin{itemize}
  \item Leitung der technischen Strategie für eine KI-gestützte Plattform im Bereich Immobilienanalyse \& Standortintelligenz.
  \item Entwicklung eines webbasierten Dashboards (Flutter Web, Firebase, Stripe, OSM APIs).
  \item Aufbau einer skalierbaren Datenpipeline, einschließlich Geodaten-Verarbeitung und Machine-Learning-Modulen.
  \item Leitung eines internationalen Entwicklerteams (Kanada, Australien, Deutschland).
  \item Verantwortung für Architektur, Qualitätssicherung \& technische Entscheidungen.
\end{itemize}

% ================== AUSBILDUNG ==================
\cvsection{AUSBILDUNG}

\textbf{MSc Autonomy Technologies | FAU Erlangen-Nürnberg} \hfill \textit{Erlangen, Deutschland | In Arbeit}

Fokus: Autonome Systeme, Robotik, Regelungstechnik, KI, eingebettete Systeme, Mensch-Maschine-Interaktion.

Wesentliche Kompetenzen:
\begin{itemize}
  \item Modellierung \& Simulation (MATLAB, Simulink)
  \item Model Predictive Control (MPC)
  \item ROS2-basierte Navigation
  \item Multi-Agent Systems
  \item Sensorfusion \& Echtzeitdatenverarbeitung
\end{itemize}

Zusätzlich vertiefe ich meine Kenntnisse in modellbasierter Entwicklung, Systemarchitektur und der Integration von KI-Methoden in regelungs- und robotiknahe Anwendungen. Dabei arbeite ich praxisnah an Teamprojekten, in denen robuste Softwarestrukturen, saubere Schnittstellen und reproduzierbare Ergebnisse im Fokus stehen.

\textbf{B.Sc. Mechatronik \& Automatisierungstechnik | Ain Shams University} \hfill \textit{Ägypten | 2022}

Gesamtnote: 1{,}8 (Deutsch).

Graduation Project (A+): Entwicklung eines intelligenten CRS Catalyst-5 Manipulators mit Kollisionsschutz (ML-basierte Bilderkennung in Python).

% ================== PAGE BREAK ==================
\clearpage

% ================== TECH STACK ==================
\cvsection{TECH STACK -- KEY SKILLS}

\textbf{Programmiersprachen:} Python (Experte), C/C++, Java, C\#, Dart/Flutter, MATLAB/Simulink, SQL, JavaScript

\textbf{Frameworks \& Tools:} ROS2, TensorFlow, PyTorch, Firebase, Google Cloud, Docker, Git/GitLab, Power BI, Power Automate

\textbf{Embedded \& Control:} RTOS, CAN, UART/I2C/SPI, MCU-Programmierung, Regelungstechnik, MPC, Automatisierungstechnik

\textbf{Software Engineering:} CI/CD, Microservices, Cloud-Architektur, Datenmodellierung, API-Design

\textbf{CAD/Mechanik:} SolidWorks, Fusion 360, Grundlagen der FEM

% ================== PRAKTIKA ==================
\cvsection{PRAKTIKA}

\textbf{P\&G School of Future Leaders}
\begin{itemize}
  \item Fokus: Leadership, Soft Skills, Projektmanagement, Teamkommunikation.
\end{itemize}

\textbf{Siemens Best-in-Class Internship | Digitale Industrie}
\begin{itemize}
  \item Fokus: Drives, Automatisierung, CNC, SPS-Programmierung, industrielle Kommunikation.
  \item Einblicke in Industrie 4.0 und moderne Automatisierungsstrategien.
\end{itemize}

\textbf{Valeo Embedded Academy}
\begin{itemize}
  \item Embedded C, Software-Architektur, Unit Testing, Automotive Protokolle (CAN).
\end{itemize}

% ================== PROJEKTE ==================
\cvsection{RELEVANTE PROJEKTERFAHRUNG}

\textbf{Autonomer Feuerwehrroboter}
\begin{itemize}
  \item Halbautonome Navigation (Raspberry Pi + Sensorfusion).
  \item Objekterkennung über Kamerasysteme.
  \item Echtzeitsystem zur Lokalisierung von Flammenquellen.
\end{itemize}

\textbf{Industrielle Mini-Produktionslinie}
\begin{itemize}
  \item Entwicklung eines 5-stufigen Systems (Materialzufuhr, Transport, Montage, Sortierung, Demontage).
  \item SPS-Logik \& embedded Steuerung.
\end{itemize}

\textbf{Aufzugssteuerung (ATmega32)}
\begin{itemize}
  \item Zustandsautomat, Echtzeitsteuerung, Sicherheitsmodus (Feuerfall).
\end{itemize}

\textbf{C-basiertes Zahlungssystem}
\begin{itemize}
  \item Vollständige Systemarchitektur (Server, Terminal, Kartenmodul).
\end{itemize}

% ================== FOOTER ==================
\vfill
\begin{tabularx}{\textwidth}{@{}X X@{}}
  \textbf{Ort, Datum} & \hfill \textbf{Unterschrift}\\[-2pt]
  Erlangen, 30.12.2025 & \hfill {\fontfamily{pzc}\selectfont\Large Mohannad Atta}
\end{tabularx}

\end{document}
